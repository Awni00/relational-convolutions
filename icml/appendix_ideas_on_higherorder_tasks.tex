\section{Some initial ideas on higher-order relational tasks}

As noted in the discussion, the tasks considered in this paper are solvable by modeling second-order relations at most. One of the main innovations of the relational convolutions architecture over existing relational architectures is its compositionality and ability to model higher-order relations. An important direction of future research is to test the architecture's ability to model hierarchical relations of increasingly higher order. Constructing such benchmarks is a non-trivial task which requires careful thought and consideration. This was outside the scope of this paper, but we provide an initial discussion here which may be useful for constructing such benchmarks in future work.

\textbf{Propositional logic.} Consider evaluating boolean logic formula such as, 
\begin{equation*}
    x_1 \land \paren{(x_2 \lor x_3) \land \paren{(\lnot x_3 \land x_4) \lor (x_5 \land x_6 \land x_7)}}.
\end{equation*}
Evaluating this logical expression (in this form) requires iteratively grouping objects and computing the relations between them. For instance, we begin by computing the relation within $g_1 = (x_3, x_4)$ and the relation within $g_2 = (x_5, x_6, x_7)$, then we compute the relation between the groups $g_1$ and $g_2$, etc. For a task which involves logical reasoning of this hierarchical form, one might imagine the group attention in RelConvNet learning the relevant groups and the relational convolution operation computing the relations within each group. Taking inspiration from logical reasoning with such hierarchical structure may lead to interesting benchmarks of higher-order relational representation.

\textbf{Sequence modeling.} In sequence modeling (e.g., language modeling), modeling the relations between objects is usually essential. For example, syntactic and semantic relations between words are crucial to parsing language. Higher-order relations are also important, capturing syntactic and semantic relational features across different locations in the text and across multiple length-scales and layers of hierarchy~\citep[see for example some relevant work in linguistics][]{frank2012hierarchical,rosario2002descent}. The attention matrix in Transformers can be thought of as implicitly representing relations between tokens. It is possible that composing Transformer layers also learns hierarchical relations. However, as shown in this work and previous work on relational representation, Transformers have limited efficiency in representing relations. Thus, incorporating relational convolutions into Transformer-based sequence models may yield meaningful improvements in the relational aspects of sequence modeling. One way to do this is by cross-attending to a the sequence of relational objects produced by relational convolutions, each of which summarizes the relations within a group of objects at some level of hierarchy.

\textbf{Set embedding.} The objective of set embedding is to map a collection of objects to a euclidean vector which represents the important features of the objects in the set~\citep{zaheer2017deep}. Depending on what the set embedding will be used for, it may need to represent a combination of object-level features and relational information, including perhaps relations of higher order. A set embedder which incorporates relational convolutions may be able to generate representations which summarize relations between objects at multiple layers of hierarchy.

\textbf{Visual scene understanding.} In a visual scene, there are typically several objects with spatial, visual, and semantic relations between them which are crucial for parsing the scene. The CLEVR benchmark on visual scene understanding~\citep{johnson2017clevr} was used in early work on relational representation~\citep{santoroSimpleNeural2017}. In more complex situations, the objects in the scene may fall into natural groupings, and the spatial, visual, and semantic relations between those \textit{groups} may be important for parsing a scene (e.g., objects forming larger components with functional dependence determined by the relations between them). Integrating relational convolutions into a visual scene understanding system may enable reasoning about such higher-order relations.