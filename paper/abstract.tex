\begin{abstract}
    A new architecture is proposed for learning representations of hierarchical relational features. Beginning with embeddings of objects, a ``multi-head relation'' module takes a sequence of embedded objects as input and produces a tensor that represents the relations between each pair of objects. A ``relational convolution''' layer then transforms the relation tensor into a sequence of new objects, each describing the relations within some group of objects at the previous layer. Graphlet filters, analogous to filters in convolutional neural networks, represent a template of relations against which the relation tensor is compared at each grouping. The architecture also proposes ways in which `soft groups' are learned as a function of temporal, feature, or contextual information. We present the motivation and details of the architecture, together with a set of experiments to demonstrate how relational convolutional neural networks can provide an effective framework for modeling relational tasks that have hierarchical structure.
\end{abstract}