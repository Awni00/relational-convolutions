
\section{Discussion}\label{sec:discussion}

\awni{[TODO]}

\textbf{Summary of strengths of RelConvNet architecture.}

\textit{Parameter-efficiency and parameter sharing.} Analogous to CNNs for image-processing, allows for learned `filters' to applied across different groups of objects to compare against templates of learned relational patterns. This accelerates learning and enables improved generalization.

The relational convolution operation we propose here is a `convolution' in the sense that we take a patch of the relation tensor, given by a group, and compare the relations within it to a template graphlet filter via an appropriately-defined inner product. This is similar to how, in a convolutional neural network, we would take a patch of an image and compare it to an image filter. Another similarity to CNNs is that the same graphlet filters are used for all groupings, thus enabling weight-sharing. This is similar to how the same convolutional filters are used for all image patches in a CNN. This makes our framework quite efficient when it comes to the number of parameters and, hopefully, makes learning easier.

\textit{Grouping and learning higher-order relations}

One important advantage of the relational convolution operation is its interpretability. The filters $\bm{f} = (f_1, \ldots, f_{n_f})$ are each a particular pattern of relations between $s$ objects. Furthermore, each object in the relational convolution $R \ast \bm{f}$ represents the degree to which the relations in the group $g$ match the patterns in each filter.


\textbf{Future work.} 

\textit{Evaluating RelConvNet on relational tasks which more explicitly rely on higher-order relations}.

\textit{Considerations when $m$ is large. How effective are learned groupings, random samples of groupings, etc.}

\textit{An evaluation of the learned grouping operations.}

\textit{Evaluating RelConvNet in more realistic settings as a module embedded in a larger architecture}. E.g., in a reinforcement learning setting. \texttt{We can mention something like codenames or other relational games. Set embeddings are another possible benchmark (though we need to find tasks which are relational, ideally compositionally-relational.)}

\textit{Composability with GNNs}. E.g., learn higher-order relational objects along with relation tensor between them to form an annotated graph then apply a GNN to it. \awni{Not sure what kinds of tasks would benifit from this.}
