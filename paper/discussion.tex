
\section{Discussion}\label{sec:discussion}

\awni{[TODO]}

\textbf{Summary of strengths of RelConvNet architecture.}

\textit{Parameter-efficiency and parameter sharing.} Analogous to CNNs for image-processing, allows for learned `filters' to applied across different groups of objects to compare against templates of learned relational patterns. This accelerates learning and enables improved generalization.

\textit{Grouping and learning higher-order relations}


\textbf{Future work.} 

\textit{Evaluating RelConvNet on relational tasks which more explicitly rely on higher-order relations}.

\textit{Considerations when $m$ is large. How effective are learned groupings, random samples of groupings, etc.}

\textit{An evaluation of the learned grouping operations.}

\textit{Evaluating RelConvNet in more realistic settings as a module embedded in a larger architecture}. E.g., in a reinforcement learning setting. \texttt{We can mention something like codenames or other relational games. Set embeddings are another possible benchmark (though we need to find tasks which are relational, ideally compositionally-relational.)}

\textit{Composability with GNNs}. E.g., learn higher-order relational objects along with relation tensor between them to form an annotated graph then apply a GNN to it. \awni{Not sure what kinds of tasks would benifit from this.}
