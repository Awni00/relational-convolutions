\section{Experiments}\label{sec:experiments}

The \textit{relational games} dataset was contributed as a benchmark for relational reasoning by~\citep{shanahanExplicitlyRelationalNeural}. It consists of a family of binary classification tasks for identifying abstract relational rules between a set of objects represented as images. The objects consist of three sets of simple geometric shapes, referred to as ``pentominoes'', ``hexominoes'', and ``stripes'' (\Cref{fig:relational_games_objects}). The objects are arranged in a $3 \times 3$ grid. Each relational game task corresponds to some relationship between the objects (see~\Cref{tab:relational_games_tasks} and~\Cref{fig:relational_games_tasks}), and the target is to classify whether the relationship holds among a given sequence of objects or not.

In our experiments, we evaluate out-of-distribution generalization by training all models on the pentominoes objects and evaluating on the hexominoes and stripes objects. The input to all models is presented as a sequence of $9$ objects, each represented as a $12 \times 12 \times 3$ RGB image. In all models, the objects are processed independently by a CNN with a shared architecture. This produces a sequence of $9$ $d$-dimensional vectors. The sequence of objects is then passed to the relation-processing component of the model. The results are then flattened and passed through an MLP with a shared architecture to produce the final prediction. We compare four models: a relational convolution model (abbreviated RelConvNet), CoRelNet~\citep{kergNeuralArchitecture2022}, PrediNet~\citep{shanahanExplicitlyRelationalNeural}, and a Transformer~\citep{vaswani2017attention}. The architectural details of each model are described in~\Cref{tab:architectures}.

The pentominoes split of the relational games dataset is used for training. We hold out 1000 samples for validation (during training) and 5000 samples for testing (after training), and use the rest as the training set. The hexominoes and stripes splits are used to test out-of-distribution generalization after training. For each task and model, we train the model for 50 epochs, while tracking training and validation loss and accuracy. We use the categorical cross-entropy loss and the Adam optimizer with learning rate $0.001$, $\beta_1 = 0.9, \beta_2 = 0.999, \epsilon = 10^{-7}$. We use a batch size of 512. For each model and task, we run 5 trials with different random seeds.

\begin{table}[ht]
    \centering
    \begin{tabular}{p{0.2\textwidth}p{0.7\textwidth}}
    \toprule
    Task              & Description                                                                                                                                                                                             \\ \midrule
    \texttt{same}              & Two random cells out of nine are occupied by an object. They are the ``same'' if they have the same color, shape, and orientation (i.e., identical image)                                               \\\hline
    \texttt{occurs}            & The top row contains one object and the bottom row contains three objects. The ``occurs'' relationship holds if at least one of the objects in the bottom row is the same as the object in the top row. \\\hline
    \texttt{xoccurs}           & Same as occurs, but the relationship holds if exactly one of the objects in the bottom row is the same as the object in the top row.                                                                    \\\hline
    \texttt{between}           & The grid is occupied by three objects in a line (horizontal or vertical). The ``between'' relationship holds if the outer objects are the same.                                                         \\\hline
    \texttt{row match pattern} & The first and third rows of the grid are occupied by three objects each. The ``match pattern'' relationship holds if the relation pattern in each row is the same (e.g., AAA, AAB, ABC, etc.)           \\ \bottomrule
\end{tabular}

    \caption{Relational games tasks.}
    \label{tab:relational_games_tasks}
\end{table}

\begin{figure}
    \centering
    \begin{subfigure}[t]{0.37\textwidth}
        \centering
        \includegraphics[width=\textwidth]{figs/relational_games_objects.pdf}
        % \caption{Examples of objects from each split.}
        \label{fig:relational_games_objects}
    \end{subfigure}
    \begin{subfigure}[t]{0.62\textwidth}
        \centering
        \includegraphics[width=\textwidth]{figs/relational_games_tasks.pdf}
        % \caption{Examples of problem instances for each task. The top row is an example where the relation holds and the bottom row is an example where the relation does not hold.}
        \label{fig:relational_games_tasks}
    \end{subfigure}
    \caption{Relational games dataset. \textbf{Left} Examples of objects from each split. \textbf{Right} Examples of problem instances for each task. The top row is an example where the relation holds and the bottom row is an example where the relation does not hold.}
    \label{fig:figurelabel}
\end{figure}

\begin{table}[ht]
    \centering
    \input{figs/experiments/architectures_table.tex}
    \caption{Model architectures.}
    \label{tab:architectures}
\end{table}

\begin{table}
    \centering
    \begin{tabular}{llll}
\toprule
              &                           &     Hexos Accuracy &   Stripes Accuracy \\
Task & Model &                    &                    \\
\midrule
same & CoRelNet &  $0.988 \pm 0.006$ &  $0.724 \pm 0.112$ \\
              & PrediNet &  $0.990 \pm 0.004$ &  $0.983 \pm 0.007$ \\
              & Transformer &  $0.997 \pm 0.001$ &  $0.993 \pm 0.004$ \\
              & RelConvNet &  $0.989 \pm 0.002$ &  $0.974 \pm 0.003$ \\
              & RelConvNet (Temporal G) &  $0.981 \pm 0.002$ &  $0.965 \pm 0.007$ \\
              & RelConvNet (Feature G) &  $0.985 \pm 0.001$ &  $0.978 \pm 0.006$ \\
              & RelConvNet (Contextual G) &  $0.981 \pm 0.002$ &  $0.969 \pm 0.012$ \\\hline
occurs & CoRelNet &  $0.992 \pm 0.004$ &  $0.518 \pm 0.012$ \\
              & PrediNet &  $0.907 \pm 0.020$ &  $0.775 \pm 0.046$ \\
              & Transformer &  $0.881 \pm 0.015$ &  $0.724 \pm 0.021$ \\
              & RelConvNet &  $0.980 \pm 0.001$ &  $0.880 \pm 0.015$ \\
              & RelConvNet (Temporal G) &  $0.983 \pm 0.001$ &  $0.920 \pm 0.012$ \\
              & RelConvNet (Feature G) &  $0.788 \pm 0.112$ &  $0.747 \pm 0.099$ \\
              & RelConvNet (Contextual G) &  $0.951 \pm 0.004$ &  $0.830 \pm 0.044$ \\\hline
xoccurs & CoRelNet &  $0.980 \pm 0.007$ &  $0.606 \pm 0.035$ \\
              & PrediNet &  $0.872 \pm 0.036$ &  $0.810 \pm 0.028$ \\
              & Transformer &  $0.867 \pm 0.017$ &  $0.753 \pm 0.031$ \\
              & RelConvNet &  $0.967 \pm 0.001$ &  $0.946 \pm 0.006$ \\
              & RelConvNet (Temporal G) &  $0.963 \pm 0.006$ &  $0.939 \pm 0.012$ \\
              & RelConvNet (Feature G) &  $0.760 \pm 0.107$ &  $0.754 \pm 0.103$ \\
              & RelConvNet (Contextual G) &  $0.942 \pm 0.020$ &  $0.901 \pm 0.029$ \\\hline
between & CoRelNet &  $0.995 \pm 0.001$ &  $0.582 \pm 0.063$ \\
              & PrediNet &  $0.978 \pm 0.006$ &  $0.950 \pm 0.019$ \\
              & Transformer &  $0.986 \pm 0.003$ &  $0.961 \pm 0.010$ \\
              & RelConvNet &  $0.991 \pm 0.001$ &  $0.988 \pm 0.002$ \\
              & RelConvNet (Temporal G) &  $0.992 \pm 0.001$ &  $0.979 \pm 0.004$ \\
              & RelConvNet (Feature G) &  $0.993 \pm 0.001$ &  $0.981 \pm 0.003$ \\
              & RelConvNet (Contextual G) &  $0.992 \pm 0.002$ &  $0.972 \pm 0.012$ \\\hline
match pattern & CoRelNet &  $0.942 \pm 0.011$ &  $0.581 \pm 0.026$ \\
              & PrediNet &  $0.710 \pm 0.040$ &  $0.658 \pm 0.053$ \\
              & Transformer &  $0.627 \pm 0.005$ &  $0.591 \pm 0.006$ \\
              & RelConvNet &  $0.961 \pm 0.015$ &  $0.870 \pm 0.041$ \\
              & RelConvNet (Temporal G) &  $0.979 \pm 0.003$ &  $0.967 \pm 0.002$ \\
              & RelConvNet (Feature G) &  $0.976 \pm 0.003$ &  $0.958 \pm 0.009$ \\
              & RelConvNet (Contextual G) &  $0.935 \pm 0.047$ &  $0.896 \pm 0.060$ \\
\bottomrule
\end{tabular}

    \caption{Out-of-distribution Generalization results on relational games. We report means $\pm$ standard error of mean over 5 trials.}
\end{table}

\begin{figure}
    \begin{subfigure}{0.5\textwidth}
        \centering
        \includegraphics[width=\textwidth]{figs/experiments/stripes_acc.pdf}
        \caption{OoD generalization on Stripes objects}
    \end{subfigure}
    \begin{subfigure}{0.5\textwidth}
        \centering
        \includegraphics[width=\textwidth]{figs/experiments/hexos_acc.pdf}
        \caption{OoD generalization on Hexos objects}
    \end{subfigure}
    \caption{OoD generalization. Error bars are a bootstrap 95\% confidence interval. 5 trials. bar height is mean.}\label{fig:ood_generalization}
\end{figure}