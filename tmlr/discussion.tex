
\section{Discussion}\label{sec:discussion}
% \textbf{Summary.} 
\subsection*{Summary}
In this paper, we proposed a compositional architecture and framework for learning hierarchical relational representations via a novel relational convolution operation. The relational convolution operation we propose here is a `convolution' in the sense that it considers a patch of the relation tensor, given by a group of objects, and compares the relations within it to a template graphlet filter via an appropriately-defined inner product. This is analogous to convolutional neural networks, where an image filter is compared against different patches of the input image. Moreover, we propose an attention-based mechanism for modeling useful groupings of objects in order to maintain scalability. By alternating inner product relation layers and relational convolution layers, we obtain an architecture that naturally models hierarchical relations.

% Since the same graphlet filters are used for all groupings, the relational convolution operation implements a form of \textit{parameter-sharing} which yields improved sample-efficiency and generalization. Another important feature of the relational convolution operation is its \textit{interpretability}. The graphlet filters $\bm{f} = (f_1, \ldots, f_{n_f})$ are each a particular pattern of relations between $s$ objects. Each object in the output of a relational convolution $R \ast \bm{f}$ represents the degree to which the relations in the group $g$ match the patterns in each filter.

%\aanote{This part is new.}
\subsection*{Discussion on relational inductive biases}
In our experiments, we observed that general-purpose sequence models like the Transformer struggle to learn tasks that involve relational reasoning in a data-efficient way. The relational inductive biases of RelConvNet, CoRelNet, and PrediNet result in significantly improved performance on the relational games tasks. These models each implement different kinds of relational inductive biases, and are each designed with different motivations in mind. For example, PrediNet's architecture is loosely inspired by the structure of predicate logic, but can be understood as ultimately producing representations of pairwise difference relations, with pairs of objects selected by an attention operation. CoRelNet is a minimal relational architecture that consists of computing an $n \times n$ inner product similarity matrix followed by a softmax normalization. RelConvNet, our proposed architecture, provides further flexibility across several dimensions. Like CoRelNet, it models relations as inner products of feature maps, but it achieves greater representational capacity by learning multi-dimensional relations through multiple learned feature maps or filters. More importantly, the relational convolutions operation enables learning higher-order relations between groups of objects. This is in contrast to both PrediNet and CoRelNet, which are limited to pairwise relations. Our experiments show that the inductive biases of RelConvNet result in improved performance in relational reasoning tasks. In particular, the \textit{SET} task, where RelConvNet was the only model able to generalize non-trivially, demonstrates the necessity for explicit inductive biases that support learning hierarchical relations.

% \textbf{Limitations and future work.} 
\subsection*{Limitations and future work}
The tasks considered here are solvable by modeling only second-order relations at most. We observe that the relational convolutional networks architecture saturates the relational games benchmark of~\citet{shanahanExplicitlyRelationalNeural}. While the ``contains set'' task demonstrates a sharp separation between relational convolutional networks and existing baselines, this task too only involves second-order relations.
% , and does not fully test the abilities of the framework. 
A more thorough evaluation of this architecture, and future architectures for modeling hierarchical relations, would require the development of new benchmark tasks and datasets that involve a larger number of objects and higher-order relations. This is a subtle and non-trivial task that we leave for future work.

The experiments considered here are synthetic relational tasks designed for a controlled evaluation. In more realistic settings, we envision relational convolutional networks as modules embedded in a broader architecture. For example, a relational convolutional network can be embedded into an RL agent to enable performing tasks involving relational reasoning. Similarly, relational convolutions can perhaps be integrated into general-purpose sequence models, such as Transformers, to enable improved relational reasoning while retaining the generality of the architecture.